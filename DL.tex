% file: DL.tex
% vim:fileencoding=utf-8:fdm=marker:ft=tex
%
% Copyright © 2022 R.F. Smith <rsmith@xs4all.nl>
% SPDX-License-Identifier: MIT
% Created: 2022-01-03T22:05:38+0100
% Last modified: 2022-01-03T22:45:06+0100

%% Fill in the sender details
\newcommand{\sendername}{S. Ender}
\newcommand{\senderaddress}{Nowhere street 42}
\newcommand{\senderpostcode}{9999 ZZ}
\newcommand{\sendertown}{NOWHERE}
%% Fill in the recipient details.
\newcommand{\recipientname}{R. Ecipient}
\newcommand{\recipientaddress}{Somehere street 1a}
\newcommand{\recipientpostcode}{9876 AA}
\newcommand{\recipientkix}{9876AA1XA}
\newcommand{\recipienttown}{SOMEWHERE}

% Below this line, no changes should be necessary.
\documentclass[tikz=true]{standalone} % {{{1
% The tikz=true option makes every tikzpicture into a separate page.
% For XeLaTeX.
\usepackage[quiet,no-math]{fontspec}
\defaultfontfeatures{Ligatures=TeX}
\setmainfont{TeX Gyre Heros}
\newfontface\kix{KIX Barcode}
\font\kixfont=kix
\def\kixcode#1{\kix #1}

\begin{document} % {{{1
\begin{tikzpicture}
    [align=left,anchor=south west]
    \draw[color=white] (0,0) rectangle (22.9,11.4);
    \node[font=\scriptsize,anchor=north west] at (0.6,10.9) {%
        \sendername{} \senderaddress{} \senderpostcode\space\space \sendertown};
    %\node[rectangle,minimum width=9cm,minimum height=3.5cm,draw=blue] at (2,2.4) {%
    \node[rectangle,minimum width=9cm,minimum height=3.5cm] at (2,2.4) {%
        \recipientname\\
        \recipientaddress\\
        \recipientpostcode\space\space \recipienttown\\[2mm]
        \fontsize{10}{10}
        \kixcode{\recipientkix}
    };
    \special{pdf: put @thispage <</Rotate 90>>}
\end{tikzpicture}
\end{document}
